%----------------------------------------------------------------------------
\chapter{Bevezetés}
\label{sec:introducton}
%----------------------------------------------------------------------------

Manapság egyre több technológia jelenik meg, aminek alapja az NLP (Natural Language Processing). Ezek a technológiák nagy része nem lenne megvalósítható szemantikai elemzés nélkül. A szemantikai elemzés célja, hogy egy nyers szövegből, vagy beszédhangból előállítsa annak a szemantikai reprezentációját. Ez a reprezentáció egy irányított gráf is lehet, amit ha a mondat szintaktikai szerkezetét reprezentáló fákból állítunk elő, akkor a teljes feladat felfogható egy gráf-transzformációként. 

Bár szemantikai elemzésre számos Deep Learning-es megoldás létezik, ezek pontatlansága nagy igényt teremt egy analitikus mély szemantikai elemzési módszerre. A gráf transzformációs megközelítés ígéretes eredményeket mutatott fel, mint például a Stanford Parser, ami TRegEx-ek segítségével végzi el tiszta analitikus módon a transzformációkat.
Több formalizmus is létezik a transzformációk leírására, mint például a HRG (hyperedge-replacement grammar) vagy az IRTG(interpreted regular tree grammars). Jelenleg is egy ezekkel kapcsolatos kutatás folyik az Automatizálási és Alkalmazott Informatikai Tanszéken.

A kutatás során az Alto(Algebraic Language Toolkit)-val dolgoztunk, ami a jelenlegi leghatékonyabb környezet IRTG-k futtatására. Ugyanakkor a kutatásnak állandó gátját jelenti, hogy az IRTG még egy fejletlen nyelvtan és nehezen átlátható; és az Alto-ból is hiányoznak fontos funkcionalitások. A problémán sokat enyhítene, ha az IRTG szabályokat RegEx-ek segítségével is meglehetne hivatkozni.

Szakdolgozatom keretében egy templatelésre alkalmas nyelvet fejlesztettem ki, ami a Slime fantázianévre hallgat. Segítségével az IRTG nyelvtanokat tömörebben és átláthatóbban lehet definiálni. Mivel az Alto java-ban készül, a nyelv Kotlinban készül ANTLRv4 segítségével. Még nincs teljesen kifejlődve, de a feladathoz szükséges megoldásokat tartalmazza. Ilyen például a template definiálás, egymásba ágyazás, RegEx-xel hivatkozás és sok egyéb. Teljes formájában egy univerzális bővítmény lesz, ami bármely nyelv vagy szöveg felett használható.
 	 	 	
A dolgozat a következőképpen épül fel: 
\begin{itemize}
\item Az 2. fejezetben bemutatom a téma nyelvészeti vonatkozásait mutatom be, és a projektet, ami kapcsán készül.

\begin{itemize}
\item Az 1. alfejezetben mutatom be a szintaxis fogalmát, és annak bevett reprezentációit, avagy a szintaktikai fákat és az UD gráfokat.
\item A 2. alfejezetben mutatom be a szemantikai elemzés fogalmát, és a 4lang-ot.
Arra is kitérek, hogy mik a különbésgek az UD és a 4lang között.
\item A 3. alfejezetben a projekt célját mutatom be, ami kapcsán a szakdolgozat készül. 
\item A 4. alfejezetben mutatom be azt az eszközt, nyelvet és algebrákat, amivel dolgozunk a projekt során.
\item Az 5. alfejezetben mutatom be az eszköz és a nyelv hiányosságait, amik enyhítése a szakdolgozat termékének a célja.
\item Az 6. alfejezetben mutatom be a jelenlegi megoldásainkat a probléma orvoslására.
\end{itemize} 

\item Az 3. fejezetben mutatom be magát a nyelvet.

\begin{itemize}
\item Az 1. alfejezetben mutatom be a szakdoga termékét jelentő nyelvet, a Slime-ot.
\item A 2. alfejezetben ismertetem a tervezés szempontjait, amiken a design döntések többsége nyugszik.
\item A 3. alfejezetben ismertetek más megoldásokat a problémára, amik a nyelv megszerkesztéséhez is alapul szolgáltak.
\item A 4. alfejezetben mutatom be magának a Slime-nak a részletes működését és szintaxisát.
\item Az 5. alfejezetben ismertetem az implementáció mögötti technológiákat és megoldásokat nagy vonalakban.
\item Az 6. alfejezetben mutatom be a nyelv fejlesztésének eddigi és ez utáni ütemtervét.
Az ambíciók demonstrációjának céljával betekintést nyújtok minden tervben lévő funkcionalitásra.
\item Az 7. alfejezetben példákkal szolgálok a Slime felhasználására több alárendelt nyelvvel is.
\end{itemize}

\end{itemize} 
